\documentclass{21kuur}

\title{UURIMISTÖÖ VORMISTAMINE \latex TARKVARAGA TALLINNA 21. KOOLI NÕUETE JÄRGI}
\author{Karl Kask}
\klass{XI klass}
\juhendaja{Siim Luha}

\begin{document}
\maketitle
\tableofcontents

\newpage
\chapter*{SISSEJUHATUS}
\addcontentsline{toc}{chapter}{SISSEJUHATUS}
Peale mu algse teemavaliku kinnitamist võttis minuga ühendust kooli infotehnoloog ja väitis, et sellist uurimistööd pole koolil vaja ning soovitas mulle alternatiivi. Kuna väljapakutu minus erilist huvi ei tekitanud, otsustasin endale uue teema välja mõelda. Peale mitmeid vestlusi oma vennaga, kes on varem uurimistööid LaTeX tarkvaraga vormistanud, sain oma lõpliku uurimistöö teema sõnastatud.
\\\\Uurimistöö vormistamine võib olla keeruline ja aeganõudev osa töö koostamisest mitmeil põhjuseil. Esiteks on vormistamise nõuded erinevatel koolidel erinevalt määratletud. Seega ei saa informatsiooni vormistamise kohta teistest allikatest, kui kooli juhendist või juhendajalt. Teiseks võivad nõuded olla vastukäivad ja lõplik kindel määratlus võib olla aeganõudev. Kolmandaks võib kõikide nõuete täpne jälgimine ja reaalne kehtestamine olla aeglane ja liigselt tähelepanu nõudev protsess. Pikaaegne vormistamise periood võib negatiivselt mõjuda töö sisule ja kvaliteedile.
\\\\Antud uurimistöö eesmärk on tutvustada LaTeX tarkvara ajalugu, sisu ja eeliseid uurimistöö vormistamisel. Innustada õpilasi kasutama LaTeX tarkvara oma uurimistöö vormistamisel, et nende töö kvaliteet ja sisu ei langeks vormistamise arvelt. 
\\\\Kaks uurimisküsimust on püstitatud uurimistöö eesmärgi täitmiseks. Mis on LaTeX tarkvara? Kuidas vormistada uurimistööd Tallinna 21. Kooli nõuete järgi LaTeX tarkvaraga? Millised on LaTeX tarkvara positiivsed ja negatiivsed küljed võrreldes enamlevinud tekstiredaktoritega?
\newpage

\chapter{TEOREETILINE TAUST}
TeX tarkvara on Donald Knuth'i loodud tavateksti kui ka muud visuaalset infot sisaldavate dokumentide loomise vahend. Knuth iseloomustas TeX tarkvara, kui imeilustae raamatute loomise vahendit - eriti, kui raamat sisaldab palju matemaatikat (Mittlebach \& Goossens, 2004: 1). 1990-ndate algul lõpetas Knuth tarkvara stabiilsuse huvides TeX-i arendamise. 1980-ndatel alustas Leslie Lamport LaTeX tarkvara arendust. LaTeX on dokumentide ettevalmistussüsteem, mis kasutab märgenduskeelt ja TeX programmi. LaTeX-i vahendid dokumendi loomise automatiseerimiseks teevad selle kasutamise lihtsamaks kui TeX-i kasutamise. LaTeX on tarkvarana kujunenud üheks peamiseks TeX-i kasutamise vahendiks. LaTeX-i filosoofia seisneb autorite vabadusel olla keskendunud sellele, millest nad kirjutavad ja mitte olla häiritud sellest, milline on kirjatöö visuaalne presentatsioon. LaTeX tarkvara kasutatakse memorandumite, nii äri kui ka isiklike kirjade, ajalehtede, artiklite ja raamatute vormistamiseks (Mittlebach \& Goossens, 2004: 1). LaTeX on põhiline keeruliste tabelite ja matemaatiliste valemite kujutamismeetod (Kopka \& Daly, 2012: 7). LaTeX tarkvara kasutatakse paljudes erinevates eluvaldkondades: nii teaduses, ajakirjanduses kui ka kunstis (Kopka \& Daly, 2012: 3). LaTeX on läbinisti vabavara. Selle arendus on lubatud avalikkuse kätte, tingimusel, et uue lahenduse nimi on muudetud (Kopka \& Daly, 2012: 3). 
\\\\LaTeX-i süsteem on struktureeritud pakettidesse, mille hulk sõltub dokumendi iseloomust ja keerukusest. Pakette on võimalik ka ise juurde luua ja sellega LaTeX-i laiendada. LaTeX dokumenti valmistades kirjeldab autor struktuuri märksõnadega nagu peatükk, osa, tabel, joonis, valem jms. LaTeX hoolitseb teksti ja muude elementide paigutuse ja peatükkide, valemite, tabelite ja piltide automaatse nummerdamise ning kasutatud kirjanduse viidete ning sisukorra eest.

\section{Miks LaTeX?}
LaTeX suurim erinevus teiste tekstiredaktoritega seisneb loogilises disainis. Enamlevinud tekstiredaktorid põhinevad visuaalsel disainil, mistõttu neid nimetatakse "WYSIWYG"(what you see is what you get) programmideks. Visuaalse disaini atraktiivsus seisneb kasutaja võimaluses muuta oma dokumendi välimust töölaual otseselt. Neid programme on iseloomustatud ka lausega "what you see is all you've got", millest tuleb välja ka nende programmide suurim nõrkus (Lamport, 1994: 7). Loogilise disaini võimsus seisneb disaini piiramatuses ja mitmete samasuguste elementide üheaegses muutmises. 

\chapter{METOODIKA}
Oma uurimistöö käigus uurin TeX ja LaTeX tarkvara ajalugu, kasutamisalasid ning plusse ja miinuseid. Oma uurimistöö praktilises osas arendan välja toimiva lahenduse uurimistöö vormistamiseks LaTeX tarkvaras ja kirjeldan üksikasjalikult iga detaili ja probleemi, mis töö käigus ette võivad sattuda. Toon välja mitmeid tähtsamaid koodinäiteid.

\chapter{PRAKTILINE OSA - KLASSI LOOMINE}
\section{Üldine vormistus}
\section{Tiitelleht}
\section{Pildid, graafikud, tabelid}

\chapter{UURIMISTÖÖ VORMISTAMINE \latex TARKVARAGA}

\chapter{TULEMUSED JA ANALÜÜS}
\newpage
\chapter*{KOKKUVÕTE}
\addcontentsline{toc}{chapter}{KOKKUVÕTE}
\newpage
\chapter*{KASUTATUD MATERJALID}
\addcontentsline{toc}{chapter}{KASUTATUD MATERJAL}
Kopka, H., Daly, P. W. (2012). Guide to LaTeX. 4. tr. Boston: Addison-Wesley
\\\\Lamport, L. (1994). LaTeX: A Document Preparation System. 2. tr. USA: Addison-Wesley
\\\\Mittelbach, F., Goossens, M. (2004). The LaTeX Companion. 2. tr. Massachusetts: Addison-Wesley
\\

\newpage
\pagenumbering{gobble}
\chapter*{Lisa 1}
\addcontentsline{toc}{chapter}{LISA 1}

\end{document}