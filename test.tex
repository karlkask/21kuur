\documentclass{21kuur}

\usepackage[estonian]{babel}

\begin{document}

\title{21.kooli uurimistöö vormistamine LaTeX abil}
\author{Minu Nimi}
\date{Oktoober 2013}
\maketitle

\chapter*{Sissejuhatus}
Siia tuleb sissejuhatus. Tuleb kasutada tärni, et ei tuleks pealkirja ette numbrit. 


\chapter{Lorem Ipsum}

\section{Mis on Lorem Ipsum?}
Lorem Ipsum on lihtsalt proovitekst, mida kasutatakse printimis- ja ladumistööstuses. See on olnud tööstuse põhiline proovitekst juba alates 1500. aastatest, mil tundmatu printija võttis hulga suvalist teksti, et teha trükinäidist. Lorem Ipsum ei ole ainult viis sajandit säilinud, vaid on ka edasi kandunud elektroonilisse trükiladumisse, jäädes sealjuures peaaegu muutumatuks. See sai tuntuks 1960. aastatel Letraset'i lehtede väljalaskmisega, ja hiljuti tekstiredaktoritega nagu Aldus PageMaker, mis sisaldavad erinevaid Lorem Ipsumi versioone.

\subsection{Mis on Lorem Ipsum? - subsection}
Lorem Ipsum on lihtsalt proovitekst, mida kasutatakse printimis- ja ladumistööstuses. See on olnud tööstuse põhiline proovitekst juba alates 1500. aastatest, mil tundmatu printija võttis hulga suvalist teksti, et teha trükinäidist. Lorem Ipsum ei ole ainult viis sajandit säilinud, vaid on ka edasi kandunud elektroonilisse trükiladumisse, jäädes sealjuures peaaegu muutumatuks. See sai tuntuks 1960. aastatel Letraset'i lehtede väljalaskmisega, ja hiljuti tekstiredaktoritega nagu Aldus PageMaker, mis sisaldavad erinevaid Lorem Ipsumi versioone.

\section{Kust see pärineb?}
Vastupidiselt tavaarusaamale, Lorem Ipsum ei ole lihtsalt suvaline tekst. See sai alguse osakesest klassikalisest ladina kirjandusest, mis on pärit aastast 45 eKr, olles seega rohkem kui kaks tuhat aastat vana. Virginia Hampden-Sydney College'i ladina keele professor Richard McClintock otsis Lorem Ipsumi lõigust ühe ebamääraseima ladinakeelse sõna, 'consecteur', ja vaadates läbi sõna viited klassikalises kirjanduses, leidis Lorem Ipsumi vaieldamatu allika. Lorem Ipsum on tulnud Cicero 45 aastat eKr kirjutatud raamatu "De Finibus Bonorum et Malorum" ("Hüvede ja pahede piiridest") osadest 1.10.32 ja 1.10.33. See raamat on uurimus eetikateooriast, mis oli renessansi ajal väga populaarne. Lorem Ipsumi esimene rida, "Lorem ipsum dolor sit amet...", tuleb reast osas 1.10.32.

Standardne tükk Lorem Ipsumit, mida kasutatakse 1500. aastatest on huvilistele allpool välja toodud. Osad 1.10.32 ja 1.10.33 Cicero "De Finibus Bonorum et Malorumist" on avaldatud nende täpsel originaalsel kujul koos H. Rackami 1914. aasta ingliskeelse tõlkega.


\end{document}